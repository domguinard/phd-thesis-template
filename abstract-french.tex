\chapter*{R\'{e}sum\'{e}}
\begin{otherlanguage}{frenchb}
L'int\'{e}gration du digital avec le r\'{e}el reste l'une des pr\'{e}occupations principales de l'informa-tique ubiquitaire et pervasive. De plus, les r\'{e}cents d\'{e}veloppements en informatique embarqu\'{e}e ont pour cons\'{e}quence un d\'{e}ploiement croissant d'\stsf{}. Nous d\'{e}finissons les \stsf{} (appel\'{e}s \sts{} dans cette th\`{e}se) comme des objets du monde r\'{e}el dou\'{e}s d'une capacit\'{e} de communication. Parmi les domaines d'application de ces objets, on peut citer: les r\'{e}seaux de capteurs d\'{e}ploy\'{e}s dans nos villes modernes, les rendant plus intelligentes et adaptatives ou la domotique permettant \`{a} nos nouveaux t\'{e}l\'{e}viseurs, radio-r\'{e}veils, frigidaires ou cadres \`{a} photos de nous rendre la vie plus facile et d'optimiser notre communication ou consommation d'\'{e}nergie. De fa\c{c}on similaire, l'industrie b\'{e}n\'{e}ficie de robots et de machines de plus en plus intelligents et les biens de consommation sont \'{e}quip\'{e}s d'\'{e}tiquettes \'{e}lectroniques ou de code-barres li\'{e}s \`{a} des sources d'information virtuelles permettant de nouveaux cas d'utilisation.

Forte de l'engouement pour ces nouveau syst\`{e}mes, la recherche dans le domaine de l'Internet des objets explore de nouvelles mani\`{e}res de connecter ces objets ensemble. Afin de faciliter ces connections, la recherche et l'industrie ont propos\'{e} plusieurs protocoles de communication. Toutefois, bien que ces protocoles facilitent la communication bas-niveau, les \stsf{} forment encore des \^{i}lots isol\'{e}s les uns des autres au niveau applicatif. Par cons\'{e}quent, la cr\'{e}ation d'applications pour des \stsf{} reste presque exclusivement accessible \`{a} des sp\'{e}cialistes du domaine de l'embarqu\'{e} tout comme la cr\'{e}ation de services composites utilisant des \stsf{}. Pour contrer cette complexit\'{e}, plusieurs plateformes orient\'{e}es services proposent une architecture d'int\'{e}gration. Toutefois, bien qu'impl\'{e}ment\'{e}es avec succ\`{e}s sur quelques appareils et machines, ces plateformes ne sont souvent pas compatibles entre elles et leur complexit\'{e} les rend difficiles d'acc\`{e}s aux novices. 

En revanche, l'Internet se pr\'{e}sente comme un tr\`{e}s bon exemple de r\'{e}seau global d'ordina-teurs h\'{e}t\'{e}rog\`{e}nes int\'{e}gr\'{e}s avec succ\`{e}s. Au-dessus de l'Internet, le Web quant \`{a} lui illustre comment un petit nombre de standards ouverts et relativement simples facilitent la construction d'applications complexes tout en pr\'{e}servant une certaine efficacit\'{e}. De plus, la capacit\'{e} du Web \`{a} supporter la cr\'{e}ation d'applications composites et interop\'{e}rables ainsi que sa disponibilit\'{e} pour une large palette d'appareils (p.ex., PC, portables, t\'{e}l\'{e}phones, set-top boxes, consoles de jeu, etc.) en font un candidat id\'{e}al pour une plateforme d'int\'{e}gration universelle. En effet, les sites Web ne sont plus de simples pages mais de v\'{e}ritables services qui peuvent \^{e}tre r\'{e}utilis\'{e}s en combinaison avec d'autres sites afin de cr\'{e}er dynamiquement de nouvelles applications s'ex\'{e}cutant en ligne et dont les clients sont de natures diverses. 

Dans cette th\`{e}se nous utilisons le Web et ses technologies \'{e}mergentes comme base pour une plateforme applicative int\'{e}grant les \stsf{}. Au sein de l'architecture applicative du Web des objets, nous proposons quatre couches permettant de simplifier la cr\'{e}ation d'applications. La premi\`{e}re couche traite de \important{l'acc\`{e}s aux \stsf{}}. Nous y dissertons de l'adaptation et de l'impl\'{e}mentation dans le monde des objets des principes architecturaux du Web. En particulier nous \'{e}tudions l'utilisation du \quote{Representational State Transfer} (REST). Nous \'{e}tendons l'architecture REST en proposant certaines adaptations afin de prendre en compte les contraintes des \stsf{}. A titre d'exemple, nous \'{e}tudions l'utilisation de passerelles et proposons des mod\`{e}les de communication en temps r\'{e}el. 

Dans la deuxi\`{e}me couche nous \'{e}tudions la \important{recherche et la localisation des objets}. Dans un Web peupl\'{e} de milliards d'objets, comment pouvons-nous retrouver celui fournissant le service le plus adapt\'{e} \`{a} notre application? Afin de r\'{e}pondre \`{a} cette question nous proposons un mod\`{e}le l\'{e}ger de m\'{e}ta-donn\'{e}es que les moteurs de recherche peuvent interpr\'{e}ter. De plus, nous impl\'{e}mentons un syst\`{e}me de registre permettant d'effectuer des recherche de services en fonction du contexte particulier des clients et des \stsf{}.

Le Web des objets tel que nous le pr\'{e}sentons dans cette th\`{e}se promouvoit un r\'{e}seau global et ouvert d'\stsf{}. Pourtant, il est peu probable que nous souhaitions laisser l'acc\`{e}s libre \`{a} tous nos objets au reste du monde. Dans la troisi\`{e}me couche nous adressons ce probl\`{e}me en proposant une infrastructure permettant le \important{partage} d'objets sur le Web. Cette infrastructure utilise les r\'{e}seaux sociaux afin de permettre un protocole de partage facilement utilisable et bas\'{e} sur nos connections personnelles, cr\'{e}ant ainsi un Web social des objets.

Notre but principal lorsque nous proposons d'amener les objets au plus proche du Web est de faciliter leur utilisation dans des applications composites. Tout comme les aficionados de la technologie et du Web cr\'{e}ent facilement des \quote{mashups} (c\`{a}d. des applications l\'{e}g\`{e}res et dynamiques utilisant plusieurs services du Web), ils devraient pouvoir en faire de m\^{e}me avec les \stsf{}. En cons\'{e}quence, la troisi\`{e}me couche traite de la composition de services et introduit la notion de \important{mashups physiques}. Nous y proposons une plateforme logicielle construite comme une extension d'un gestionnaire de processus et offrant des \'{e}l\'{e}ments de langage permettant de cr\'{e}er des \'{e}diteurs de mashup pour les \stsf{}.

Finalement, afin de d'\'{e}valuer l'architecture et les outils propos\'{e}s, nous nous attardons sur deux types d'\stsf{}. Tout d'abord nous consid\'{e}rons les r\'{e}seaux de capteurs environnementaux. Les b\'{e}n\'{e}fices de l'utilisation du Web des objets y sont test\'{e}s de fa\c{c}on empirique par le biais de plusieurs prototypes, de fa\c{c}on quantitative \`{a} l'aide d'\'{e}valuations de performances, puis de fa\c{c}on qualitative par le biais d'\'{e}tudes avec des d\'{e}veloppeurs utilisant ces approches. Ensuite, nous \'{e}tudions le cas des syst\`{e}mes d'identification par ondes radio (RFID) et proposons une structure permettant d'amener les donn\'{e}es et appareils RFID sur le Web. Nous \'{e}valuons les performances de cette structure et illustrons ses b\'{e}n\'{e}fices par le biais de plusieurs prototypes.

Mises ensemble, les contributions de cette th\`{e}se proposent les fondations du futur Web des objets: un r\'{e}seau d'objets et de services global et interop\'{e}rable au-dessus duquel des applications peuvent \^{e}tre cr\'{e}\'{e}es avec aisance. Cette th\`{e}se permet donc de diminuer le foss\'{e} qui existe encore entre notre monde de tous les jours et le monde virtuel.
\end{otherlanguage}