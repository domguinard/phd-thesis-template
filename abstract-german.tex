\chapter*{Kurzfassung}
\begin{otherlanguage}{ngerman}
Die Integration von digitalen Artefakten mit der physischen Welt ist ein zentrales Anliegen des Pervasive Computing. J\"{u}ngste Entwicklungen im Bereich der eingebetteten Systeme haben dazu gef\"{u}hrt, dass wir in unserem t\"{a}glichen Leben immer \"{o}fter mit \quote{smarten} Dingen -- vernetzten, digital angereicherten Ger\"{a}ten -- interagieren. Unsere St\"{a}dte werden durch drahtlose Sensor- und Aktuatornetze intelligenter und erm\"{o}glichen kontextsensitives Verhalten. Neue Ger\"{a}te wie intelligente Fernseher, Wecker, K\"{u}hlschr\"{a}nke oder digitale Bilderrahmen machen unsere Wohnungen und H\"{a}user energieeffizienter und unser Leben angenehmer. Die Industrie profitiert von zunehmend intelligenteren Maschinen und Robotern. Allt\"{a}gliche Objekte, die mit Funkchips markiert oder mit Strichcodes versehen sind, werden um virtuelle Informationsquellen erweitert und bieten neue Gesch\"{a}fts-m\"{o}glichkeiten.

Als Folge dieser Entwicklung wird in den letzten Jahren im Rahmen des \newterm{Internet der Dinge} nach M\"{o}glichkeiten gesucht, smarte Dinge miteinander zu vernetzen. Um das Verbinden von Ger\"{a}ten zu vereinfachen, haben Forschung und Industrie eine Reihe von Niedrigenergie-Kommunkationsprotokollen konzipiert und standardisiert. Eine Folge dieser Entwicklungen war jedoch, dass sich auf der Anwendungsebene unter den verbundenen Ger\"{a}ten immer mehr kleine, unvereinbare Inseln bildeten. Das Erstellen von Anwendungen f\"{u}r smarte Dinge ist heute eine anspruchsvolle Aufgabe, die Fachwissen \"{u}ber jede einzelne Plattform erfordert. Dies erschwert die Integration von vernetzten Alltagsgegenst\"{a}nden in ger\"{a}te\"{u}bergreifenden Anwendungen. Um dieser Entwicklung entgegenzuwirken, erschienen immer mehr Integrationsarchitekturen die zwar zum Teil erfolgreich eingesetzt werden, jedoch meistens nicht miteinander kompatibel sind. Ihre Komplexit\"{a}t und der Mangel an unterst\"{u}tzenden Werkzeugen f\"{u}hren dazu, dass sie nur von einer kleinen Gruppe von Experten verwendet werden k\"{o}nnen und damit ihr Nutzen f\"{u}r die Erstellung von innovativen Anwendungen bisher eher begrenzt ist.

Das Internet ist ein \"{u}berzeugendes Beispiel f\"{u}r ein skalierbares weltweites Computernetz, in dem heterogene Hardware- und Softwareplattformen ohne Integrationsprobleme zusammenarbeiten. Des Weiteren zeigt das World Wide Web, wie durch die Nutzung von vergleichbar einfachen und offenen Standards hochflexible Systeme gebaut werden k\"{o}nnen, w\"{a}hrend die Effizienz und Skalierbarkeit des Internet weiterhin gew\"{a}hrleistet sind. Aufgrund seiner breiten Verf\"{u}gbarkeit auf verschiedenen Ger\"{a}ten (z.B. PCs, Laptops, Mobiltelefone, Set-Top-Boxen, Spielkonsolen etc.) und seiner hohen Flexibilit\"{a}t ist das Web ein hervorragender Kandidat f\"{u}r eine universelle Integrationsplattform. Webseiten bieten neben ihrem klassischen Inhalt auch Programmierschnittstellen, die von anderen Web-Ressourcen verwendet werden k\"{o}nnen, um innovative Anwendungen in der Cloud zu erstellen, auf die von Desktops oder mobilen Computern aus zugegriffen werden kann.

In dieser Arbeit schlagen wir das Web of Things als eine vierschichtige Applikationsintegrationsarchitektur vor, die die Entwicklung von Anwendungen mit smarten Dingen vereinfacht. Zun\"{a}chst wenden wir uns dem Problem der \important{Zug\"{a}nglichkeit} von Ger\"{a}ten zu und schlagen die Umsetzung der Prinzipien, die das Herzst\"{u}ck des Web bilden (z.B. Representational State Transfer, REST), auf smarten Dingen vor.

Des Weiteren haben wir die REST-Architektur erweitert, um die speziellen Anforderungen der physischen Welt -- etwa die Notwendigkeit f\"{u}r dom\"{a}nenspezifische Proxies oder f\"{u}r Echtzeit-Kommunikation -- zu ber\"{u}cksichtigen. Ausserdem betrachten wir die Frage der \important{Auffindbarkeit}: Wie k\"{o}nnen in einem Netz von Milliarden von smarten Dingen diejenigen Ger\"{a}te, die f\"{u}r eine bestimmte Anwendung ben\"{o}tigte Dienste anbieten, gefunden werden, und wie kann mit ihnen interagiert werden? Zur L\"{o}sung dieser Probleme schlagen wir ein leichtgewichtiges, f\"{u}r Suchmaschinen lesbares, Metadaten-Format vor, das mit einer Web-basierten Auffindungs- und Suchinfrastruktur zusammenarbeitet, die den besonderen Kontext von smarten Dingen ber\"{u}cksichtigt.

Obwohl das Web of Things ein offenes Netz von physischen Objekten unterst\"{u}tzt, ist es unwahrscheinlich, dass der Zugriff auf smarte Dinge in Zukunft f\"{u}r jedermann uneingeschr\"{a}nkt m\"{o}glich sein wird. Aus diesem Grund bauen wir auf der dritten Ebene eine Infrastruktur, die soziale Netzwerke verwendet, um das gemeinsame Nutzen von smarten Dingen zu erm\"{o}glichen. Wir zeigen, wie ein solches \newterm{social Web of Things} die \important{kollektive Verwendung} von physischen Artefakten auf benutzerfreundliche Art und Weise erm\"{o}glicht.

Das Hauptziel dieser Arbeit ist die Verwendung des Web, um die Integration von smarten Dingen in kollaborativen Anwendungen zu vereinfachen. Dadurch erm\"{o}glichen wir Webentwicklern und anderen Personen mit guten Computerkenntnissen, innovative, auf smarten Alltagsdingen basierende, Anwendungen zu entwickeln -- so, wie sie heute Web 2.0-Mashups (leichtgewichtige, ad hoc aus verschiedenen Web-Diensten zusammengesetzte Applikationen) erstellen. Um dieses Ziel zu erreichen, stellen wir auf einer weiteren Ebene (\important{Composition Layer}) das Konzept von physischen Mashups vor. Wir schlagen ausserdem eine Softwareplattform vor, die als Erweiterung eines quelloffenen Workflow-Systems konzipiert wurde und Grundkonstrukte bereitstellt, um Mashup-Umgebungen f\"{u}r das Web of Things zu erstellen.

Wir verwenden zwei verschiedene Arten von smarten Dingen, um unsere vorgeschlagene Architektur und die dazugeh\"{o}rigen Werkzeuge zu testen: Zuerst betrachten wir drahtlose Sensorknoten, insbesondere solche, die f\"{u}r Umwelt- und Energie\"{u}berwachung eingesetzt werden. Wir bewerten die mit dem Einsatz unserer Architektur zusammenh\"{a}ngenden Effekte empirisch in verschiedenen Prototypen, quantitativ \"{u}ber Leistungsmessungen und qualitativ mithilfe von mehreren Entwicklern, die unsere Systeme bei der Erstellung von mobilen Applikationen und Webanwendungen einsetzen. Um ausserdem zu verstehen, wie das Web of Things die Entwicklung von intelligenten Gesch\"{a}ftsanwendungen vereinfachen kann, betrachten wir Systeme zur automatischen Identifizierung von Gegenst\"{a}nden und schlagen ein System vor, mit dem RFID-Daten in das World Wide Web und globale RFID-Informationssysteme in die Cloud integriert werden k\"{o}nnen. Wir demonstrieren das Leistungsverhalten dieses Systems anhand von verschiedenen Prototypen.

Zusammengefasst liefern unsere Beitr\"{a}ge ein \"{O}kosystem von Bausteinen f\"{u}r ein globales Netz von interoperablen smarten Dingen, die das Erstellen von ger\"{a}te\-\"{u}bergreifenden Anwendungen vereinfachen, welche die Kluft zwischen der virtuellen und der physischen Welt \"{u}berbr\"{u}cken.
\end{otherlanguage}